%% Generated by Sphinx.
\def\sphinxdocclass{report}
\documentclass[letterpaper,10pt,english]{sphinxmanual}
\ifdefined\pdfpxdimen
   \let\sphinxpxdimen\pdfpxdimen\else\newdimen\sphinxpxdimen
\fi \sphinxpxdimen=.75bp\relax
\ifdefined\pdfimageresolution
    \pdfimageresolution= \numexpr \dimexpr1in\relax/\sphinxpxdimen\relax
\fi
%% let collapsible pdf bookmarks panel have high depth per default
\PassOptionsToPackage{bookmarksdepth=5}{hyperref}

\PassOptionsToPackage{booktabs}{sphinx}
\PassOptionsToPackage{colorrows}{sphinx}

\PassOptionsToPackage{warn}{textcomp}
\usepackage[utf8]{inputenc}
\ifdefined\DeclareUnicodeCharacter
% support both utf8 and utf8x syntaxes
  \ifdefined\DeclareUnicodeCharacterAsOptional
    \def\sphinxDUC#1{\DeclareUnicodeCharacter{"#1}}
  \else
    \let\sphinxDUC\DeclareUnicodeCharacter
  \fi
  \sphinxDUC{00A0}{\nobreakspace}
  \sphinxDUC{2500}{\sphinxunichar{2500}}
  \sphinxDUC{2502}{\sphinxunichar{2502}}
  \sphinxDUC{2514}{\sphinxunichar{2514}}
  \sphinxDUC{251C}{\sphinxunichar{251C}}
  \sphinxDUC{2572}{\textbackslash}
\fi
\usepackage{cmap}
\usepackage[T1]{fontenc}
\usepackage{amsmath,amssymb,amstext}
\usepackage{babel}



\usepackage{tgtermes}
\usepackage{tgheros}
\renewcommand{\ttdefault}{txtt}



\usepackage[Bjarne]{fncychap}
\usepackage{sphinx}

\fvset{fontsize=auto}
\usepackage{geometry}


% Include hyperref last.
\usepackage{hyperref}
% Fix anchor placement for figures with captions.
\usepackage{hypcap}% it must be loaded after hyperref.
% Set up styles of URL: it should be placed after hyperref.
\urlstyle{same}


\usepackage{sphinxmessages}
\setcounter{tocdepth}{1}



\title{PLSLib}
\date{May 09, 2023}
\release{0.1}
\author{Marco Tabacman}
\newcommand{\sphinxlogo}{\vbox{}}
\renewcommand{\releasename}{Release}
\makeindex
\begin{document}

\ifdefined\shorthandoff
  \ifnum\catcode`\=\string=\active\shorthandoff{=}\fi
  \ifnum\catcode`\"=\active\shorthandoff{"}\fi
\fi

\pagestyle{empty}
\sphinxmaketitle
\pagestyle{plain}
\sphinxtableofcontents
\pagestyle{normal}
\phantomsection\label{\detokenize{index::doc}}


\sphinxAtStartPar
\sphinxstylestrong{PLSLib} is a Python and R library implementing the various algorithms detailed in the book \sphinxstyleemphasis{Partial LeastSquares Regression and Related Dimension Reduction Methods} by R. Dennis Cook and Liliana Forzani, available \sphinxhref{about:blank}{here}.

\sphinxstepscope


\chapter{NIPALS}
\label{\detokenize{NIPALS:nipals}}\label{\detokenize{NIPALS::doc}}
\sphinxAtStartPar
Nonlinear Iterative Partial Least Squares {[}REFERENCE HERE{]}.
\index{nipals (class in NIPALS.nipals)@\spxentry{nipals}\spxextra{class in NIPALS.nipals}}

\begin{fulllineitems}
\phantomsection\label{\detokenize{NIPALS:NIPALS.nipals.nipals}}
\pysigstartsignatures
\pysigline{\sphinxbfcode{\sphinxupquote{class\DUrole{w,w}{  }}}\sphinxcode{\sphinxupquote{NIPALS.nipals.}}\sphinxbfcode{\sphinxupquote{nipals}}}
\pysigstopsignatures
\sphinxAtStartPar
Orthogonal weights: \(W^T_q W_q = I_q\)

\sphinxAtStartPar
Envelope connection: \(\mathrm{span}(W_q) = \mathcal{E}_{\Sigma_X}(\mathcal{B})\), the \(\Sigma_X\)\sphinxhyphen{}envelope of \(\mathcal{B} \mathrel{\vcenter{:}}= \mathrm{span}(\beta)\).

\sphinxAtStartPar
Score matrix \(S_d\): These are traditional computational intermediaries,
although they are not needed in the computation of \(\hat{\beta}_{\mathrm{npls}}\).

\sphinxAtStartPar
Algorithm \(\mathbb{N}\): This is an instance of Algorithm \(\mathbb{N}\) discussed in \S{}1.5.3.

\sphinxAtStartPar
PLS1 v. PLS2: Algorithm is applicable for PLS1 or PLS2 fits; See \S{}3.8.

\end{fulllineitems}

\index{fit() (in module NIPALS.nipals.nipals)@\spxentry{fit()}\spxextra{in module NIPALS.nipals.nipals}}

\begin{fulllineitems}
\phantomsection\label{\detokenize{NIPALS:NIPALS.nipals.nipals.fit}}
\pysigstartsignatures
\pysiglinewithargsret{\sphinxcode{\sphinxupquote{NIPALS.nipals.nipals.}}\sphinxbfcode{\sphinxupquote{fit}}}{\sphinxparam{\DUrole{n,n}{self}}, \sphinxparam{\DUrole{n,n}{X}}, \sphinxparam{\DUrole{n,n}{Y}}, \sphinxparam{\DUrole{n,n}{q}}, \sphinxparam{\DUrole{n,n}{version}\DUrole{o,o}{=}\DUrole{default_value}{\textquotesingle{}sample\textquotesingle{}}}}{}
\pysigstopsignatures
\sphinxAtStartPar
Fit this model to the training data \sphinxtitleref{X}, \sphinxtitleref{Y} using \sphinxtitleref{q} dimensions.
\begin{quote}\begin{description}
\sphinxlineitem{Parameters}\begin{itemize}
\item {} 
\sphinxAtStartPar
\sphinxstyleliteralstrong{\sphinxupquote{X}} (\sphinxstyleliteralemphasis{\sphinxupquote{array\sphinxhyphen{}like}}) \textendash{} Predictor of shape (\sphinxtitleref{n\_samples}, \sphinxtitleref{p\_features})

\item {} 
\sphinxAtStartPar
\sphinxstyleliteralstrong{\sphinxupquote{Y}} (\sphinxstyleliteralemphasis{\sphinxupquote{array\sphinxhyphen{}like}}) \textendash{} Response of shape (\sphinxtitleref{n\_samples}, \sphinxtitleref{r\_features})

\item {} 
\sphinxAtStartPar
\sphinxstyleliteralstrong{\sphinxupquote{q}} (\sphinxstyleliteralemphasis{\sphinxupquote{int}}) \textendash{} Value between \sphinxtitleref{1} and \sphinxtitleref{p\_features}. The number of projections used.

\item {} 
\sphinxAtStartPar
\sphinxstyleliteralstrong{\sphinxupquote{version}} (\sphinxstyleliteralemphasis{\sphinxupquote{str}}) \textendash{} either ‘sample’ or ‘population’, defaults to ‘sample’

\end{itemize}

\sphinxlineitem{Returns}
\sphinxAtStartPar
Nothing.

\end{description}\end{quote}

\end{fulllineitems}

\index{transform() (in module NIPALS.nipals.nipals)@\spxentry{transform()}\spxextra{in module NIPALS.nipals.nipals}}

\begin{fulllineitems}
\phantomsection\label{\detokenize{NIPALS:NIPALS.nipals.nipals.transform}}
\pysigstartsignatures
\pysiglinewithargsret{\sphinxcode{\sphinxupquote{NIPALS.nipals.nipals.}}\sphinxbfcode{\sphinxupquote{transform}}}{\sphinxparam{\DUrole{n,n}{self}}, \sphinxparam{\DUrole{n,n}{X}}}{}
\pysigstopsignatures
\sphinxAtStartPar
Transform data using the NIPALS algorithm. Must run {\hyperref[\detokenize{NIPALS:NIPALS.nipals.nipals.fit}]{\sphinxcrossref{\sphinxcode{\sphinxupquote{NIPALS.nipals.nipals.fit()}}}}} before running this function.
\begin{quote}\begin{description}
\sphinxlineitem{Parameters}
\sphinxAtStartPar
\sphinxstyleliteralstrong{\sphinxupquote{X}} (\sphinxstyleliteralemphasis{\sphinxupquote{array\sphinxhyphen{}like}}) \textendash{} Predictor of shape (\sphinxtitleref{n\_samples}, \sphinxtitleref{p\_features})

\sphinxlineitem{Returns}
\sphinxAtStartPar
The \(W\) and \(\beta\) transformed data, respectively.

\sphinxlineitem{Return type}
\sphinxAtStartPar
tuple(array\sphinxhyphen{}like, array\sphinxhyphen{}like)

\end{description}\end{quote}

\end{fulllineitems}




\renewcommand{\indexname}{Index}
\printindex
\end{document}